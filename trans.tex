\documentclass[UTF8, zihao = -4, linespread = 1.335, heading = true, fontset = none]{ctexart}
\usepackage{ujnthesis}

\begin{document}


\includepdf[pages=-]{docs/06-cover-translation.pdf}     % 外文资料翻译封面
\includepdf[pages=-]{docs/07-text-en.pdf}               % 外文资料翻译原文


%%%%%%%%%%%%%%%%%%%%%%%%%%%%%% 元数据 %%%%%%%%%%%%%%%%%%%%%%%%%%%%%%%

\transource{Journal of American Chemical Society}{2006}{128(7)}{2421-2425}
\transtitle{测试标题}
\transauther{测试作者}
\transinstitution{测试机构}
\transabstract{
    测试摘要测试摘要测试摘要测试摘要测试摘要测试摘要测试摘要测试摘要测试摘要测试摘要测试摘要测试摘要测试摘要测试摘要测试摘要
}{测试,测试,测试}

%%%%%%%%%%%%%%%%%%%%%%%%%%%%%% 元数据 %%%%%%%%%%%%%%%%%%%%%%%%%%%%%%%

\transbody
%\setfancylength
\pagestyle{ujntranslation}

%%%%%%%%%%%%%%%%%%%%%%%%%%%%%% 正文区 %%%%%%%%%%%%%%%%%%%%%%%%%%%%%%%

\section{功能测试(节标题section)}
\subsection{文字与段落(子节标题subsection)}
这是文字。
\subsubsection{段落(子小节标题subsubsection)}
这是段落。这是段落。这是段落。这是段落。这是段落。这是段落。这是段落。这是段落。这是段落。这是段落。这是段落。这是段落。这是段落。这是段落。这是段落。这是段落。这是段落。这是段落。这是段落。这是段落。

这是另一个段落。这是另一个段落。这是另一个段落。这是另一个段落。这是另一个段落。这是另一个段落。这是另一个段落。这是另一个段落。这是另一个段落。这是另一个段落。这是另一个段落。这是另一个段落。这是另一个段落。
\subsection{数学公式}
\subsubsection{行内公式}
这是简单的行内公式:$x^2+y^2=z^2$,这是复杂的行内公式:$\sum_{i=1}^n a_i=0$。
\subsubsection{行间公式}
(1)线性代数:
\begin{equation}
    \begin{split}
        \mathbf{A}^{-1} &= \frac{1}{\det(\mathbf{A})}\mathbf{A}^* \\
        \mathbf{A}^* &= \begin{bmatrix}
            \mathbf{A}_{11}^* & \mathbf{A}_{12}^* & \cdots & \mathbf{A}_{1n}^* \\
            \mathbf{A}_{21}^* & \mathbf{A}_{22}^* & \cdots & \mathbf{A}_{2n}^* \\
            \vdots & \vdots & \ddots & \vdots \\
            \mathbf{A}_{m1}^* & \mathbf{A}_{m2}^* & \cdots & \mathbf{A}_{mn}^*
        \end{bmatrix} \\
    \end{split}
\end{equation}

(2)微积分:
\begin{equation}
    \begin{split}
        \frac{d}{dx}f(x) &= \lim_{h \to 0}\frac{f(x+h)-f(x)}{h} \\
        \frac{d^2}{dx^2}f(x) &= \lim_{h \to 0}\frac{f(x+h)-2f(x)+f(x-h)}{h^2} \\
        \frac{d^n}{dx^n}f(x) &= \lim_{h \to 0}\frac{f(x+h)-f(x)-\cdots-f(x-(n-1)h)}{h^n}
    \end{split}
    \label{eq:1}
\end{equation}

(3)概率论与数理统计:
\begin{equation}
    \begin{split}
        P(A) &= \frac{\text{事件A发生的次数}}{\text{总次数}} \\
        P(A \mid B) &= \frac{P(A \cap B)}{P(B)} \\
        P(A \cap B) &= P(A)P(B \mid A) \\
        P(A \cup B) &= P(A) + P(B) - P(A \cap B)
    \end{split}
\end{equation}

\begin{equation}
    \begin{split}
        E(X) &= \sum_{i=1}^n x_iP(X=x_i) \\
        E(XY) &= \sum_{i=1}^n \sum_{j=1}^n x_iy_jP(X=x_i,Y=y_j) \\
        E(X \mid Y) &= \sum_{i=1}^n x_iP(X=x_i \mid Y=y) \\
        E(X \mid Y=y) &= \sum_{i=1}^n x_iP(X=x_i,Y=y) \\
        E(X \mid Y=y_1,y_2,\cdots,y_k) &= \sum_{i=1}^n x_iP(X=x_i,Y=y_1,Y=y_2,\cdots,Y=y_k)
    \end{split}
\end{equation}

(4)数学分析:

\begin{equation}
    \begin{split}
        \lim_{x \to a}f(x) &= L \\
        \lim_{x \to a^+}f(x) &= L \\
        \lim_{x \to a^-}f(x) &= L \\
        \lim_{x \to a^+}f(x) &= \lim_{x \to a^-}f(x) \\
        \lim_{x \to a^+}f(x) &= \lim_{x \to a^-}f(x) = L
    \end{split}
\end{equation}

(5)离散数学:
\begin{equation}
    \begin{split}
        \binom{n}{k} &= \frac{n!}{k!(n-k)!} \\
        \binom{n}{0} &= \binom{n}{n} = 1 \\
        \binom{n}{1} &= \binom{n}{n-1} = n \\
        \binom{n}{2} &= \binom{n}{n-2} = \frac{n(n-1)}{2}
    \end{split}
\end{equation}

(6)复变函数:
\begin{equation}
    \begin{split}
        \lim_{z \to \infty}f(z) &= L \\
        \lim_{z \to \infty}f(z) &= \lim_{z \to -\infty}f(z) \\
        \lim_{z \to \infty}f(z) &= \lim_{z \to -\infty}f(z) = L \\
        \lim_{z \to \infty}f(z) &= \lim_{z \to -\infty}f(z) \neq L
    \end{split}
\end{equation}
\subsection{代码块与图表}
\subsubsection{代码块}
\begin{lstlisting}[language=C]
    #include <stdio.h>
    /* hello,world */
    int main(){
        printf("Hello, World! \n"); 
        return 0;
    }
\end{lstlisting}
\subsubsection{图片}
\begin{figure}[htbp]
    \centering
    \includegraphics[scale=0.1]{figures/pikachu.jpg}
    \caption{这是图片}
    \label{fig:1}
\end{figure}
\begin{figure}[htbp]
    \centering
    \includegraphics[scale=0.12]{figures/children.jpg}
    \caption{这是图片}
    \label{fig:2}
\end{figure}
\subsubsection{表格}
\begin{table}[!htbp]
    \centering
    \caption{这是表格}
    \begin{tabular}{cccccc}
        \toprule
        序号 & 姓名 & 性别 & 年龄 & 身高/cm & 体重/kg \\
        \midrule
        1 & 张三 & M & 16 & 163 & 50 \\
        2 & 王红 & F & 15 & 159 & 47 \\
        3 & 李二 & M & 17 & 165 & 52 \\
        \bottomrule
    \end{tabular}
    \label{tab:1}
\end{table}
\subsection{交叉引用}\label{sec:1}
\subsubsection{图表引用}\label{sec:2}
图片\ref{fig:1}和表格\ref{tab:1}的交叉引用。
\subsubsection{章节引用}
章节\ref{sec:1}和章节\ref{sec:2}的交叉引用。
\subsubsection{公式引用}
公式\ref{eq:1}的交叉引用。
\section{功能测试(节标题section)}
\subsection{文字与段落(子节标题subsection)}
这是文字。
\subsubsection{段落(子小节标题subsubsection)}
这是段落。这是段落。这是段落。这是段落。这是段落。这是段落。这是段落。这是段落。这是段落。这是段落。这是段落。这是段落。这是段落。这是段落。这是段落。这是段落。这是段落。这是段落。这是段落。这是段落。

这是另一个段落。这是另一个段落。这是另一个段落。这是另一个段落。这是另一个段落。这是另一个段落。这是另一个段落。这是另一个段落。这是另一个段落。这是另一个段落。这是另一个段落。这是另一个段落。这是另一个段落。
\subsection{数学公式}
\subsubsection{行内公式}
这是简单的行内公式:$x^2+y^2=z^2$,这是复杂的行内公式:$\sum_{i=1}^n a_i=0$。
\subsubsection{行间公式}
(1)线性代数:
\begin{equation}
    \begin{split}
        \mathbf{A}^{-1} &= \frac{1}{\det(\mathbf{A})}\mathbf{A}^* \\
        \mathbf{A}^* &= \begin{bmatrix}
            \mathbf{A}_{11}^* & \mathbf{A}_{12}^* & \cdots & \mathbf{A}_{1n}^* \\
            \mathbf{A}_{21}^* & \mathbf{A}_{22}^* & \cdots & \mathbf{A}_{2n}^* \\
            \vdots & \vdots & \ddots & \vdots \\
            \mathbf{A}_{m1}^* & \mathbf{A}_{m2}^* & \cdots & \mathbf{A}_{mn}^*
        \end{bmatrix} \\
    \end{split}
\end{equation}

(2)微积分:
\begin{equation}
    \begin{split}
        \frac{d}{dx}f(x) &= \lim_{h \to 0}\frac{f(x+h)-f(x)}{h} \\
        \frac{d^2}{dx^2}f(x) &= \lim_{h \to 0}\frac{f(x+h)-2f(x)+f(x-h)}{h^2} \\
        \frac{d^n}{dx^n}f(x) &= \lim_{h \to 0}\frac{f(x+h)-f(x)-\cdots-f(x-(n-1)h)}{h^n}
    \end{split}
    \label{eq:2}
\end{equation}

(3)概率论与数理统计:
\begin{equation}
    \begin{split}
        P(A) &= \frac{\text{事件A发生的次数}}{\text{总次数}} \\
        P(A \mid B) &= \frac{P(A \cap B)}{P(B)} \\
        P(A \cap B) &= P(A)P(B \mid A) \\
        P(A \cup B) &= P(A) + P(B) - P(A \cap B)
    \end{split}
\end{equation}

\begin{equation}
    \begin{split}
        E(X) &= \sum_{i=1}^n x_iP(X=x_i) \\
        E(XY) &= \sum_{i=1}^n \sum_{j=1}^n x_iy_jP(X=x_i,Y=y_j) \\
        E(X \mid Y) &= \sum_{i=1}^n x_iP(X=x_i \mid Y=y) \\
        E(X \mid Y=y) &= \sum_{i=1}^n x_iP(X=x_i,Y=y) \\
        E(X \mid Y=y_1,y_2,\cdots,y_k) &= \sum_{i=1}^n x_iP(X=x_i,Y=y_1,Y=y_2,\cdots,Y=y_k)
    \end{split}
\end{equation}

(4)数学分析:

\begin{equation}
    \begin{split}
        \lim_{x \to a}f(x) &= L \\
        \lim_{x \to a^+}f(x) &= L \\
        \lim_{x \to a^-}f(x) &= L \\
        \lim_{x \to a^+}f(x) &= \lim_{x \to a^-}f(x) \\
        \lim_{x \to a^+}f(x) &= \lim_{x \to a^-}f(x) = L
    \end{split}
\end{equation}

(5)离散数学:
\begin{equation}
    \begin{split}
        \binom{n}{k} &= \frac{n!}{k!(n-k)!} \\
        \binom{n}{0} &= \binom{n}{n} = 1 \\
        \binom{n}{1} &= \binom{n}{n-1} = n \\
        \binom{n}{2} &= \binom{n}{n-2} = \frac{n(n-1)}{2}
    \end{split}
\end{equation}

(6)复变函数:
\begin{equation}
    \begin{split}
        \lim_{z \to \infty}f(z) &= L \\
        \lim_{z \to \infty}f(z) &= \lim_{z \to -\infty}f(z) \\
        \lim_{z \to \infty}f(z) &= \lim_{z \to -\infty}f(z) = L \\
        \lim_{z \to \infty}f(z) &= \lim_{z \to -\infty}f(z) \neq L
    \end{split}
\end{equation}
\subsection{代码块与图表}
\subsubsection{代码块}
\begin{lstlisting}[language=C]
    #include <stdio.h>
    /* hello,world */
    int main(){
        printf("Hello, World! \n"); 
        return 0;
    }
\end{lstlisting}
\subsubsection{图片}
\begin{figure}[htbp]
    \centering
    \includegraphics[scale=0.1]{figures/pikachu.jpg}
    \caption{这是图片}
    \label{fig:3}
\end{figure}
\begin{figure}[htbp]
    \centering
    \includegraphics[scale=0.12]{figures/children.jpg}
    \caption{这是图片}
    \label{fig:4}
\end{figure}
\subsubsection{表格}
\begin{table}[!htbp]
    \centering
    \caption{这是表格}
    \begin{tabular}{cccccc}
        \toprule
        序号 & 姓名 & 性别 & 年龄 & 身高/cm & 体重/kg \\
        \midrule
        1 & 张三 & M & 16 & 163 & 50 \\
        2 & 王红 & F & 15 & 159 & 47 \\
        3 & 李二 & M & 17 & 165 & 52 \\
        \bottomrule
    \end{tabular}
    \label{tab:2}
\end{table}
\subsection{交叉引用}\label{sec:3}
\subsubsection{图表引用}\label{sec:4}
图片\ref{fig:2}和表格\ref{tab:2}的交叉引用。
\subsubsection{章节引用}
章节\ref{sec:3}和章节\ref{sec:4}的交叉引用。
\subsubsection{公式引用}
公式\ref{eq:2}的交叉引用。

%%%%%%%%%%%%%%%%%%%%%%%%%%%%%% 正文区 %%%%%%%%%%%%%%%%%%%%%%%%%%%%%%%

\end{document}